\chapter{Supplemental results and analyses for Chapter 2}

\section{Plot-level analysis}

\subsection{Explanation}
In order to calculate energetic compensation and the total energy ratio, we require an estimate for the baseline values of total energy use, kangaroo rat energy use, and small granivore energy use on control plots. Estimating these baselines requires aggregating over between-plot variability among the control plots. For consistency, in the main analysis, we also aggregate across the exclosure plots and focus on treatment-level means throughout. Here, we explore the effect of between-plot variability on our analyses, to the extent possible. We used treatment-level means across control plots to calculate energetic compensation and the total energy ratio, but calculated these quantities separately for each exclosure plot, and conducted analyses including a random effect of plot. We also conducted analyses of Dipodomys and C. baileyi proportional energy use using plot-level data, again including plot as a random effect. Results were qualitatively the same as using treatment-level means.

\subsection{Compensation}

\subsubsection{Model specification and selection}

We fit linear mixed-effects models (using the $lme$ function in the R package $nlme$; \cite{pinheiro2020}) of the form $compensation = time period$ with a random effect of plot and temporal autocorrelation structure to account for autocorrelation between monthly census periods within each time period. We compared these to models without the autocorrelation structure, without the random effect, and without the term for time period. The best-fitting model included terms for time period, random effect of plot, and autocorrelation.

\begin{table}[htbp]% Fix the table captions to sit directly on the table, but figures do NOT sit directly on the figure.
     \captionof{table}{Plot-level:  Model comparison for compensation.}\label{first}
    \begin{tabularx}{\textwidth}{XX}
      \hline
      Model specification & AIC \\
      \hline
      intercept + timeperiod + plot (random effect) + autocorrelation & 	1360.207 \\
      intercept + timeperiod + plot (random effect)  & 	1680.916 \\
      intercept + timeperiod +  autocorrelation & 	1409.830 \\
      intercept + plot (random effect) + autocorrelation & 	1408.362 \\
      intercept +  plot (random effect) & 	1879.126 \\
      intercept  & 	2036.371 \\
      \hline
    \end{tabularx}
\end{table}

\subsubsection{Results}

\begin{table}[htbp]% Fix the table captions to sit directly on the table, but figures do NOT sit directly on the figure.
     \captionof{table}{Plot-level:  Coefficients from linear mixed-effects model for compensation.}\label{first}
    \begin{tabularx}{\textwidth}{XXXXXX}
      \hline
       & Value & Std.Error & DF & t-value & p-value \\
      \hline
(Intercept) & 0.3451282 &	0.1048354 &	1362 &	3.292096 &	0.0010199\\
     oera.L &	0.0653090 &	0.0373313 &	1362 &	1.749446 &	0.0804392 \\
oera.Q &	-0.2845830 &	0.0341063 &	1362 &	-8.343990 &	0.0000000 \\
      \hline
    \end{tabularx}
\end{table}


\begin{table}[htbp]% Fix the table captions to sit directly on the table, but figures do NOT sit directly on the figure.
     \captionof{table}{Plot-level:  Estimates from linear mixed-effects model for compensation.}\label{first}
    \begin{tabularx}{\textwidth}{XXXXXX}
      \hline
        Timeperiod &	emmean &	SE &	df &	lower.CL &	upper.CL\\
      \hline
1988-1997 &	0.1827673 & 0.1091842 &	3 &	-0.1647055 &	0.5302400 \\
1997-2010 &	0.5774892 &	0.1078860 &	3 &	0.2341478 &	0.9208306 \\
2010-2020 &	0.2751282 &	0.1093969 &	 3 &	-0.0730215 &	0.6232779 \\
      \hline
    \end{tabularx}
\end{table}

\begin{table}[htbp]% Fix the table captions to sit directly on the table, but figures do NOT sit directly on the figure.
     \captionof{table}{Plot-level:  Contrasts from linear mixed-effects model for compensation.}\label{first}
    \begin{tabularx}{\textwidth}{XXXXXX}
      \hline
        Comparison &	estimate &	SE &	df &	t.ratio &	p.value\\
      \hline
1988-1997 - 1997-2010	& -0.3947220 &	0.0491845	& 1362	& -8.025330	& 0.0000 \\
1988-1997 - 2010-2020	& -0.0923609 &	0.0527944	& 1362	& -1.749446	& 0.1873 \\
1997-2010 - 2010-2020	& 0.3023610	& 0.0496411	& 1362	& 6.090948	& 0.0000 \\
      \hline
    \end{tabularx}
\end{table}


\subsection{Total energy use}

\subsubsection{Model specification and selection}

As for compensation, we fit linear mixed-effects models fitting total_energy_ratio ~ time period with a random effect of plot and a temporal autocorrelation term to account for autocorrelation between monthly census periods within each timeperiod. We compared these to models without the autocorrelation term, without the random effect, and without the term for time period. The best-fitting model included terms for time period, random effect of plot, and autocorrelation.

\begin{table}[htbp]% Fix the table captions to sit directly on the table, but figures do NOT sit directly on the figure.
     \captionof{table}{Plot-level:  Model comparison for total energy use.}\label{first}
    \begin{tabularx}{\textwidth}{XX}
      \hline
      Model specification & AIC \\
      \hline
      intercept + timeperiod + plot (random effect) + autocorrelation & 	474.8558 \\
      intercept + timeperiod + plot (random effect)  & 	924.1830 \\
      intercept + timeperiod +  autocorrelation & 	507.7842 \\
      intercept + plot (random effect) + autocorrelation & 	543.5425 \\
      intercept +  plot (random effect) & 	1266.2097\\
      intercept  & 	1382.7469 \\
      \hline
    \end{tabularx}
\end{table}


\subsubsection{Results}


\begin{table}[htbp]% Fix the table captions to sit directly on the table, but figures do NOT sit directly on the figure.
     \captionof{table}{Plot-level:  Coefficients from linear mixed-effects model for total energy ratio.}\label{first}
    \begin{tabularx}{\textwidth}{XXXXXX}
      \hline
       & Value & Std.Error & DF & t-value & p-value \\
      \hline
        (Intercept) & 0.5018200	& 0.0709701	& 1362	& 7.070865 &	0.0e+00\\
        oera.L &		0.1454309 &	0.0301324 &	1362 &	4.826392 &	1.5e-06\\
        oera.Q &-0.2545852 &	0.0273660	& 1362 &	-9.302977 &	0.0e+00 \\
      \hline
    \end{tabularx}
\end{table}

Table S7. Estimates from linear mixed-effects model on total energy ratio
Timeperiod	emmean	SE	df	lower.CL	upper.CL
1988-1997	0.2950508	0.0751321	3	0.0559470	0.5341547
1997-2010	0.7096879	0.0738511	3	0.4746606	0.9447151
2010-2020	0.5007212	0.0752881	3	0.2611207	0.7403216
Table S8. Contrasts from linear mixed-effects model on total energy ratio
Comparison	estimate	SE	df	t.ratio	p.value
1988-1997 - 1997-2010	-0.4146370	0.0395736	1362	-10.477622	0.0e+00
1988-1997 - 2010-2020	-0.2056703	0.0426137	1362	-4.826392	4.6e-06
1997-2010 - 2010-2020	0.2089667	0.0398571	1362	5.242901	5.0e-07
 
Kangaroo rat proportional energy use
Model specification and selection
To compare proportional energy use across time periods, we used binomial generalized linear mixed models (using the glmer function in the R package lme4; Bates et al. 2015), which allowed us to include a random effect of plot.
For Dipodomys proportional energy use, we compared models with and without the random effect of plot and with and without a term for timeperiod. The best-fitting model included terms for timeperiod and a random effect of plot.
Table S9. Model comparison for Dipodomys proportional energy use.
Model.specification	AIC
intercept + timeperiod + plot (random effect)	1040.861
intercept + plot (random effect)	1162.470
intercept + timeperiod	1108.490
intercept	1208.081
Results
Table S10. Coefficients from GLMER on Dipodomys energy use.
Note that “oera” is the variable name for the term for time period in these analyses.
	Estimate	Std. Error	z value	Pr(>|z|)
(Intercept)	2.181163	0.1305753	16.704251	0
oera.L	-1.946096	0.2664545	-7.303670	0
oera.Q	1.124620	0.1769225	6.356572	0
Table S11. Estimates from GLMER on Dipodomys energy use.
Note that estimates are back-transformed onto the response scale, for interpretability.
Timeperiod	prob	SE	df	asymp.LCL	asymp.UCL
1988-1997	0.9823009	0.0062020	Inf	0.9701452	0.9944566
1997-2010	0.7795273	0.0183934	Inf	0.7434769	0.8155777
2010-2020	0.7797464	0.0208516	Inf	0.7388780	0.8206149
Table S12. Contrasts from GLMER on Dipodomys energy use.
Contrasts are performed on the link (logit) scale.
Comparison	estimate	SE	df	z.ratio	p.value
1988-1997 - 1997-2010	0.2027736	0.0194108	Inf	10.4464200	0
1988-1997 - 2010-2020	0.2025545	0.0217545	Inf	9.3109407	0
1997-2010 - 2010-2020	-0.0002191	0.0278048	Inf	-0.0078811	1
 
C. baileyi proportional energy use
Model specification and selection
As for kangaroo rat proportional energy use, we used a binomial generalized linear mixed effects model to compare C. baileyi proportional energy use across time periods. Because C. baileyi occurs on both control and exclosure plots, we investigated whether the dynamics of C. baileyi’s proportional energy use differed between treatment types. We compared models incorporating separate slopes, separate intercepts, or no terms for treatment modulating the change in C. baileyi proportional energy use across time periods, i.e. comparing the full set of models:
•	cbaileyi_proportional_energy_use ~ timeperiod + treatment + timeperiod:treatment
•	cbaileyi_proportional_energy_use ~ timeperiod + treatment
•	cbaileyi_proportional_energy_use ~ timeperiod
We also tested a null (intercept-only) model of no change across time periods:
•	cbaileyi_proportional_energy_use ~ 1
We compared all of these models with and without a random effect of plot.
We found that the best-fitting model incorporated a random effect of plot, and fixed effects for time period and for treatment, but no interaction between them (cbaileyi_proportional_energy_use ~ timeperiod + treatment). We therefore proceeded with this model.
Table S13. Model comparison for C. baileyi proportional energy use.
Model.specification	AIC
intercept + timeperiod + treatment + timeperiod:treatment + plot (random effect)	1021.318
intercept + timeperiod + treatment + plot (random effect)	1020.263
intercept + timeperiod + plot (random effect)	1042.758
intercept + plot (random effect)	1321.149
intercept + timeperiod + treatment + timeperiod:treatment	1166.653
intercept + timeperiod + treatment	1162.901
intercept + timeperiod	1869.097
intercept	2036.489
Results
Table S14. Coefficients from GLMER on C. baileyi energy use
Note that “oera” is the variable name for the term for time period in these analyses, and “oplottype” refers to experimental treatment.
	Estimate	Std. Error	z value	Pr(>|z|)
(Intercept)	-2.443643	0.2067789	-11.81766	0
oera.L	-1.866286	0.1530068	-12.19740	0
oplottype.L	3.265183	0.2913472	11.20719	0
Table S15. Estimates from GLMER on C. baileyi energy use
Note that estimates are back-transformed onto the response scale, for interpretability.
Timeperiod	Treatment	prob	SE	df	asymp.LCL	asymp.UCL
1997-2010	Control	0.0312856	0.0116044	Inf	0.0085414	0.0540297
1997-2010	Exclosure	0.7658194	0.0392864	Inf	0.6888195	0.8428193
2010-2020	Control	0.0023009	0.0008486	Inf	0.0006378	0.0039641
2010-2020	Exclosure	0.1893142	0.0364430	Inf	0.1178872	0.2607412
Table S16. Contrasts from GLMER on C. baileyi energy use.
Contrasts are performed on the link (logit) scale.
Comparison	Treatment	estimate	SE	df	z.ratio	p.value
1997-2010 - 2010-2020	Control	2.639326	0.2163843	Inf	12.1974	0
1997-2010 - 2010-2020	Exclosure	2.639326	0.2163843	Inf	12.1974	0

\chapter{Secondary Appendix Content}

\begin{table}[htbp]% Fix the table captions to sit directly on the table, but figures do NOT sit directly on the figure.
    \captionof{table}{A sample Table using tabularx}\label{first}
    \begin{tabularx}{6.5in}{XXX}
      \hline
      First & Second & Third \\
      \hline
      12 & 45 & 26 \\
      17 & 32 & 93 \\
      text & 51 & can be there too. \\	
      \hline
    \end{tabularx}
\end{table}


\begin{figure}[h]
    \centering
    \includegraphics[scale=.2]{graphics/theworld.png}
    \caption[Figure Caption]{This is a long caption to make sure that the caption in the list-of section is correctly single space with the blank white line between captions. }
    \label{fig:my_label}
\end{figure}

Test for second appendix file.


%\chapter{Expert Opinions}

Test for third appendix file.


