\chapter{Supplemental results and analyses for Chapter 2}

\section{Plot-level analysis}

\subsection{Explanation}
In order to calculate energetic compensation and the total energy ratio, we require an estimate for the baseline values of total energy use, kangaroo rat energy use, and small granivore energy use on control plots. Estimating these baselines requires aggregating over between-plot variability among the control plots. For consistency, in the main analysis, we also aggregate across the exclosure plots and focus on treatment-level means throughout. Here, we explore the effect of between-plot variability on our analyses, to the extent possible. We used treatment-level means across control plots to calculate energetic compensation and the total energy ratio, but calculated these quantities separately for each exclosure plot, and conducted analyses including a random effect of plot. We also conducted analyses of Dipodomys and C. baileyi proportional energy use using plot-level data, again including plot as a random effect. Results were qualitatively the same as using treatment-level means.

\subsection{Compensation}


We fit linear mixed-effects models (using the $lme$ function in the R package $nlme$; \cite{pinheiro2020}) of the form $compensation = time period$ with a random effect of plot and temporal autocorrelation structure to account for autocorrelation between monthly census periods within each time period. We compared these to models without the autocorrelation structure, without the random effect, and without the term for time period. The best-fitting model included terms for time period, random effect of plot, and autocorrelation.

\subsection{Total energy use}


As for compensation, we fit linear mixed-effects models fitting \textit{total energy ratio = time period} with a random effect of plot and a temporal autocorrelation term to account for autocorrelation between monthly census periods within each timeperiod. We compared these to models without the autocorrelation term, without the random effect, and without the term for time period. The best-fitting model included terms for time period, random effect of plot, and autocorrelation.



\subsection{\textit{Dipodomys} proportional energy use}


To compare proportional energy use across time periods, we used binomial generalized linear mixed models (using the glmer function in the R package \textit{lme4}; Bates (2015)), which allowed us to include a random effect of plot.

For \textit{Dipodomys} proportional energy use, we compared models with and without the random effect of plot and with and without a term for timeperiod. The best-fitting model included terms for timeperiod and a random effect of plot.

\subsection{\textit{C. baileyi} proportional energy use}

As for kangaroo rat proportional energy use, we used a binomial generalized linear mixed effects model to compare \textit{C. baileyi} proportional energy use across time periods. Because \textit{C. baileyi} occurs on both control and exclosure plots, we investigated whether the dynamics of \textit{C. baileyi}’s proportional energy use differed between treatment types. We compared models incorporating separate slopes, separate intercepts, or no terms for treatment modulating the change in \textit{C. baileyi} proportional energy use across time periods, i.e. comparing the full set of models:

-	\textit{c baileyi proportional energy use = timeperiod + treatment + timeperiod:treatment}

-\textit{c baileyi proportional energy use = 1}

-	\textit{c baileyi proportional energy use = timeperiod}

We also tested a null (intercept-only) model of no change across time periods:

- \textit{c baileyi proportional energy use = 1}

We compared all of these models with and without a random effect of plot.
We found that the best-fitting model incorporated a random effect of plot, and fixed effects for time period and for treatment, but no interaction between them (\textit{c baileyi proportional energy use = 1}). We therefore proceeded with this model.

\subsection{ Tables}

\begin{table}[H]% Fix the table captions to sit directly on the table, but figures do NOT sit directly on the figure.
     \captionof{table}{Plot-level:  Model comparison for compensation.}\label{first}
    \begin{tabularx}{\textwidth}{XX}
      \hline
      Model specification & AIC \\
      \hline
      intercept + timeperiod + plot (random effect) + autocorrelation & 	1360.207 \\
      intercept + timeperiod + plot (random effect)  & 	1680.916 \\
      intercept + timeperiod +  autocorrelation & 	1409.830 \\
      intercept + plot (random effect) + autocorrelation & 	1408.362 \\
      intercept +  plot (random effect) & 	1879.126 \\
      intercept  & 	2036.371 \\
      \hline
    \end{tabularx}
\end{table}


\begin{table}[H]% Fix the table captions to sit directly on the table, but figures do NOT sit directly on the figure.
     \captionof{table}{Plot-level:  Coefficients from linear mixed-effects model for compensation.}\label{first}
    \begin{tabularx}{\textwidth}{XXXXXX}
      \hline
       & Value & Std.Error & DF & t-value & p-value \\
      \hline
(Intercept) & 0.3451282 &	0.1048354 &	1362 &	3.292096 &	0.0010199\\
     oera.L &	0.0653090 &	0.0373313 &	1362 &	1.749446 &	0.0804392 \\
oera.Q &	-0.2845830 &	0.0341063 &	1362 &	-8.343990 &	0.0000000 \\
      \hline
    \end{tabularx}
\end{table}


\begin{table}[H]% Fix the table captions to sit directly on the table, but figures do NOT sit directly on the figure.
     \captionof{table}{Plot-level:  Estimates from linear mixed-effects model for compensation.}\label{first}
    \begin{tabularx}{\textwidth}{XXXXXX}
      \hline
        Timeperiod &	emmean &	SE &	df &	lower.CL &	upper.CL\\
      \hline
1988-1997 &	0.1827673 & 0.1091842 &	3 &	-0.1647055 &	0.5302400 \\
1997-2010 &	0.5774892 &	0.1078860 &	3 &	0.2341478 &	0.9208306 \\
2010-2020 &	0.2751282 &	0.1093969 &	 3 &	-0.0730215 &	0.6232779 \\
      \hline
    \end{tabularx}
\end{table}

\begin{table}[H]% Fix the table captions to sit directly on the table, but figures do NOT sit directly on the figure.
     \captionof{table}{Plot-level:  Contrasts from linear mixed-effects model for compensation.}\label{first}
    \begin{tabularx}{\textwidth}{XXXXXX}
      \hline
        Comparison &	estimate &	SE &	df &	t.ratio &	p.value \\
      \hline
        1988-1997 - 1997-2010	& -0.3947220 &	0.0491845	& 1362	& -8.025330	& 0.0000 \\
      1988-1997 - 2010-2020	& -0.0923609 &	0.0527944	& 1362	& -1.749446	& 0.1873 \\
        1997-2010 - 2010-2020	& 0.3023610	& 0.0496411	& 1362	& 6.090948	& 0.0000 \\
      \hline
     \end{tabularx}
\end{table}



\begin{table}[H]% Fix the table captions to sit directly on the table, but figures do NOT sit directly on the figure.
     \captionof{table}{Plot-level:  Model comparison for total energy use.}\label{first}
    \begin{tabularx}{\textwidth}{XX}
      \hline
      Model specification & AIC \\
      \hline
      intercept + timeperiod + plot (random effect) + autocorrelation & 	474.8558 \\
      intercept + timeperiod + plot (random effect)  & 	924.1830 \\
      intercept + timeperiod +  autocorrelation & 	507.7842 \\
      intercept + plot (random effect) + autocorrelation & 	543.5425 \\
      intercept +  plot (random effect) & 	1266.2097\\
      intercept  & 	1382.7469 \\
      \hline
    \end{tabularx}
\end{table}



\begin{table}[H]% Fix the table captions to sit directly on the table, but figures do NOT sit directly on the figure.
     \captionof{table}{Plot-level:  Coefficients from linear mixed-effects model for total energy ratio.}\label{first}
    \begin{tabularx}{\textwidth}{XXXXXX}
      \hline
       & Value & Std.Error & DF & t-value & p-value \\
      \hline
        (Intercept) & 0.5018200	& 0.0709701	& 1362	& 7.070865 &	0.0e+00\\
        oera.L &		0.1454309 &	0.0301324 &	1362 &	4.826392 &	1.5e-06\\
        oera.Q &-0.2545852 &	0.0273660	& 1362 &	-9.302977 &	0.0e+00 \\
      \hline
    \end{tabularx}
\end{table}


\begin{table}[H]% Fix the table captions to sit directly on the table, but figures do NOT sit directly on the figure.
     \captionof{table}{Plot-level:  Estimates from linear mixed-effects model for total energy ratio.}\label{first}
    \begin{tabularx}{\textwidth}{XXXXXX}
      \hline
        Timeperiod &	emmean &	SE &	df &	lower.CL &	upper.CL\\
      \hline
        1988-1997 &	0.2950508 & 0.0751321 &	3 &	0.0559470 &	0.5341547\\
        1997-2010 &	0.7096879 &		0.0738511 &	3 &		0.4746606 &	0.9447151 \\
        2010-2020 &	0.5007212 &		0.0752881 &	 3 &	0.2611207 &	0.7403216 \\
      \hline
    \end{tabularx}
\end{table}



\begin{table}[H]% Fix the table captions to sit directly on the table, but figures do NOT sit directly on the figure.
     \captionof{table}{Plot-level:  Contrasts from linear mixed-effects model for total energy ratio.}\label{first}
    \begin{tabularx}{\textwidth}{XXXXXX}
      \hline
        Comparison &	estimate &	SE &	df &	t.ratio &	p.value\\
      \hline
        1988-1997 - 1997-2010	& -0.4146370 &	0.0395736 &	1362 &	-10.477622 &	0.0e+00\\
        1988-1997 - 2010-2020	& 	-0.2056703 &	0.0426137 &	1362 &	-4.826392 &	4.6e-06 \\
        1997-2010 - 2010-2020	& 	0.2089667 &	0.0398571 &	1362 &	5.242901 &	5.0e-07\\
      \hline
    \end{tabularx}
\end{table}


\begin{table}[H]% Fix the table captions to sit directly on the table, but figures do NOT sit directly on the figure.
     \captionof{table}{Plot-level:  Model comparison for kangaroo rat proportional energy use.}\label{first}
    \begin{tabularx}{\textwidth}{XX}
      \hline
      Model specification & AIC \\
      \hline
      intercept + timeperiod + plot (random effect)  & 	1040.861\\
      intercept +  plot (random effect) & 		1162.470\\
      intercept + timeperiod &	1108.490 \\
      intercept  & 	1208.081 \\
      \hline
    \end{tabularx}
\end{table}


\begin{table}[H]% Fix the table captions to sit directly on the table, but figures do NOT sit directly on the figure.
     \captionof{table}{Plot-level:  Coefficients from GLMER on kangaroo rat energy use.}\label{first}
    \begin{tabularx}{\textwidth}{XXXXX}
     \hline
       & Estimate & Std. Error & z.value & $Pr(>|z|)$ \\
     \hline
     (Intercept) &	2.181163 &	0.1305753 &	16.704251 &	0 \\
     oera.L &	-1.946096 &	0.2664545 &	-7.303670 &	0 \\
     oera.Q &	1.124620 &	0.1769225 &	6.356572 &	0 \\
      \hline
    \end{tabularx}
\end{table}


\begin{table}[H]% Fix the table captions to sit directly on the table, but figures do NOT sit directly on the figure.
     \captionof{table}{Plot-level:  Estimates from GLMER on kangaroo rat energy use.}\label{first}
    \begin{tabularx}{\textwidth}{XXXXXX}
      \hline
Timeperiod &	prob &	SE &	df &	asymp.LCL &	asymp.UCL  \\
\hline
1988-1997 &	0.9823009 &	0.0062020 &	Inf	& 0.9701452	& 0.9944566 \\
1997-2010 &	0.7795273  &	0.0183934 &	Inf	& 0.7434769 &	0.8155777 \\
2010-2020 &	0.7797464 &	0.0208516 &	Inf	& 0.7388780 &	0.8206149 \\
\hline
    \end{tabularx}
\end{table}




\begin{table}[H]% Fix the table captions to sit directly on the table, but figures do NOT sit directly on the figure.
     \captionof{table}{Plot-level:  Contrasts from GLMER on kangaroo rat energy use.}\label{first}
    \begin{tabularx}{\textwidth}{XXXXXX}
      \hline
Comparison &	estmimate &	SE &	df &	z.ratio &	p.value \\
\hline
1988-1997 - 1997-2010	& 0.2027736	& 0.0194108	& Inf	& 10.4464200 &	0 \\
1988-1997 - 2010-2020	& 0.2025545	& 0.0217545	& Inf	& 9.3109407 &	0 \\
1997-2010 - 2010-2020	& -0.0002191	& 0.0278048	& Inf	& -0.0078811 &	1 \\
\hline
     \end{tabularx}
\end{table}


\begin{table}[H]% Fix the table captions to sit directly on the table, but figures do NOT sit directly on the figure.
     \captionof{table}{Plot-level:  Model comparison for \textit{C. baileyi} proportional energy use.}\label{first}
    \begin{tabularx}{\textwidth}{XX}
      \hline
      Model specification & AIC \\
      \hline
    intercept + timeperiod + treatment + timeperiod:treatment + plot (random effect) &	1021.318 \\
    intercept + timeperiod + treatment + plot (random effect) &	1020.263 \\
    intercept + timeperiod + plot (random effect) &	1042.758 \\
    intercept + plot (random effect) &	1321.149 \\
    intercept + timeperiod + treatment + timeperiod:treatment &	1166.653 \\
    intercept + timeperiod + treatment &	1162.901 \\
    intercept + timeperiod	& 1869.097 \\
    intercept &	2036.489 \\
      \hline
    \end{tabularx}
\end{table}


\begin{table}[H]% Fix the table captions to sit directly on the table, but figures do NOT sit directly on the figure.
     \captionof{table}{Plot-level:  Coefficients from GLMER on \textit{C. baileyi} energy use.}\label{first}
    \begin{tabularx}{\textwidth}{XXXXX}
     \hline
       & Estimate & Std. Error & z.value & $Pr(>|z|)$ \\
     \hline
     (Intercept) &	-2.443643 &	0.2067789 &	-11.81766 &	0\\
     oera.L &		-1.866286 &	0.1530068 &	-12.19740 &	0 \\
     oera.Q & 3.265183 &	0.2913472 &	11.20719 &	0\\
      \hline
    \end{tabularx}
\end{table}


\begin{table}[H]% Fix the table captions to sit directly on the table, but figures do NOT sit directly on the figure.
     \captionof{table}{Plot-level:  Estimates from GLMER on \textit{C. baileyi} energy use.}\label{first}
    \begin{tabularx}{\textwidth}{XXXXXXX}
      \hline
Timeperiod & Treatment &	prob &	SE &	df &	asymp.LCL &	asymp.UCL  \\
\hline
1997-2010 &	Control &	0.0312856 &	0.0116044 &	Inf &	0.0085414 &	0.0540297 \\
1997-2010 &	Exclosure &	0.7658194 &	0.0392864 &	Inf	& 0.6888195 &	0.8428193 \\
2010-2020 &	Control	& 0.0023009	& 0.0008486 & 	Inf &	0.0006378 &	0.0039641 \\
2010-2020 &	Exclosure & 	0.1893142 &	0.0364430 &	Inf &	0.1178872 &	0.2607412 \\
\hline
    \end{tabularx}
\end{table}




\begin{table}[H]% Fix the table captions to sit directly on the table, but figures do NOT sit directly on the figure.
     \captionof{table}{Plot-level:  Contrasts from GLMER on \textit{C. baileyi} energy use.}\label{first}
    \begin{tabularx}{\textwidth}{XXXXXXX}
      \hline
Comparison & Treatment &	estimate &	SE &	df &	z.ratio &	p.value \\
\hline
1997-2010 - 2010-2020 &	Control &	2.639326 &	0.2163843 &	Inf &	12.1974	& 0 \\
1997-2010 - 2010-2020 &	Exclosure &	2.639326 &	0.2163843 &	Inf	& 12.1974 &	0 \\
\hline
 \end{tabularx}
\end{table}

\section{Full model results}

\subsection{Compensation}

We fit a generalized least squares (of the form $compensation = timeperiod$; note that “timeperiod” is coded as “oera” throughout) using the $gls$ function from the R package $nlme$ \cite{pinheiro2020}. Because values from monthly censuses within each time period are subject to temporal autocorrelation, we included a continuous autoregressive temporal autocorrelation structure of order 1 (using the $CORCAR1$ function). We compared this model to models fit without the autocorrelation structure and without the time period term using AIC. The model with both the time period term and the autocorrelation structure was the best-fitting model via AIC, and we used this model to calculate estimates and contrasts using the package \textit{emmeans} \cite{lenth2021}.

\subsection{Total energy use ratio}

As for compensation, we fit a generalized least squares of the form \textit{total energy ratio = timeperiod}, accounting for temporal autocorrelation between monthly censuses within each time period using a continuous autoregressive autocorrelation structure of order 1. We compared this model to models fit without the timeperiod term and/or autocorrelation structure, and found the full (timeperiod plus autocorrelation) model had the best performance via AIC. We used this model for estimates and contrasts.

\subsection{\textit{Dipodomys} proportional energy use}


Proportional energy use is bounded 0-1 and cannot be fit with generalized least squares. We therefore used a binomial generalized linear model of the form \textit{Dipodomys proportional energy use = timeperiod}. We compared a model fit with a timeperiod term to an intercept-only (null) model using AIC, and found the timeperiod term improved model fit. We used this model for estimates and contrasts.

Note that we were unable to incorporate temporal autocorrelation into generalized linear models, and we prioritized fitting models of the appropirate family over accounting for autocorrelation. Due to the pronounced differences between time periods for these variables, we were comfortable proceeding without explicitly accounting for autocorrelation.

\subsection{\textit{C. baileyi} proportional energy use}


As for kangaroo rat proportional energy use, we used a binomial generalized linear model to compare \textit{C. baileyi} proportional energy use across time periods. Because \textit{C. baileyi} occurs on both control and exclosure plots, we investigated whether the dynamics of \textit{C. baileyi}’s proportional energy use differed between treatment types. We compared models incorporating separate slopes, separate intercepts, or no terms for treatment modulating the change in \textit{C. baileyi} proportional energy use across time periods, i.e. comparing the full set of models:

-	\textit{c baileyi proportional energy use = timeperiod + treatment + timeperiod:treatment}

-	\textit{c baileyi proportional energy use = timeperiod + treatment}

-	\textit{c baileyi proportional energy use = timeperiod}
We also tested a null (intercept-only) model of no change across time periods:

- \textit{c baileyi proportional energy use = 1}

We found that the best-fitting model incorporated effects for time period and for treatment, but no interaction between them (\textit{c baileyi proportional energy use = timeperiod + treatment}). We therefore proceeded with this model.

\subsection{Tables}

\begin{table}[H]% Fix the table captions to sit directly on the table, but figures do NOT sit directly on the figure.
     \captionof{table}{Model comparison for compensation.}\label{first}
    \begin{tabularx}{\textwidth}{XX}
      \hline
      Model specification & AIC \\
      \hline
      intercept + timeperiod + autocorrelation	& 69.85023 \\
intercept + autocorrelation	& 84.74902 \\
intercept + timeperiod &	157.09726 \\
intercept	& 252.74534 \\
      \hline
    \end{tabularx}
\end{table}


\begin{table}[H]% Fix the table captions to sit directly on the table, but figures do NOT sit directly on the figure.
     \captionof{table}{Coefficients from GLS model for compensation.}\label{first}
    \begin{tabularx}{\textwidth}{XXXXX}
      \hline
       & Value & Std.Error &  t-value & p-value \\
      \hline
(Intercept) &	0.3450313 &	0.0294996 &	11.696141 &	0.0000000 \\
oera.L &	0.0647933 &	0.0524103 &	1.236269	& 0.2172146 \\
oera.Q &	-0.2833553 &	0.0477359 &	-5.935890 &	0.0000000 \\
      \hline
    \end{tabularx}
\end{table}

\begin{table}[H]% Fix the table captions to sit directly on the table, but figures do NOT sit directly on the figure.
     \captionof{table}{Estimates from GLS for compensation.}\label{first}
    \begin{tabularx}{\textwidth}{XXXXXX}
      \hline
        Timeperiod &	emmean &	SE &	df &	lower.CL &	upper.CL\\
      \hline
1988-1997 &	0.1835362 &	0.0520378 &	44.11081 &	0.0786683 &	0.2884041 \\
1997-2010  &	0.5763899 &	0.0462641 &	47.37851 &	0.4833383 &	0.6694416 \\
2010-2020 &	0.2751677 &	0.0528010 &	46.75897 &	0.1689314 &	0.3814041 \\
      \hline
    \end{tabularx}
\end{table}


\begin{table}[H]% Fix the table captions to sit directly on the table, but figures do NOT sit directly on the figure.
     \captionof{table}{Contrasts from GLS for compensation.}\label{first}
    \begin{tabularx}{\textwidth}{XXXXXX}
      \hline
        Comparison &	estimate &	SE &	df &	t.ratio &	p.value \\
      \hline
1988-1997 - 1997-2010 &	-0.3928537 &	0.0689413 &	47.89422 &	-5.698378 &	0.0000 \\
1988-1997 - 2010-2020 &	-0.0916315 &	0.0741194 &	45.51740 &	-1.236269 &	0.4383 \\
1997-2010 - 2010-2020 &	0.3012222 &	0.0694989 &	49.52957 &	4.334200 &	0.0002 \\
      \hline
     \end{tabularx}
\end{table}






\begin{table}[H]% Fix the table captions to sit directly on the table, but figures do NOT sit directly on the figure.
     \captionof{table}{Model comparison for total energy use.}\label{first}
    \begin{tabularx}{\textwidth}{XX}
      \hline
      Model specification & AIC \\
      \hline
     intercept + timeperiod + autocorrelation &	-132.92138 \\
intercept + autocorrelation &	-118.15000 \\
intercept + timeperiod &	13.29396 \\
intercept &	156.85988 \\
      \hline
    \end{tabularx}
\end{table}


\begin{table}[H]% Fix the table captions to sit directly on the table, but figures do NOT sit directly on the figure.
     \captionof{table}{Coefficients from GLS for total energy ratio.}\label{first}
    \begin{tabularx}{\textwidth}{XXXXXX}
      \hline
       & Value & Std.Error  & t-value & p-value \\
      \hline
        (Intercept) &	0.5016731 &	0.0271176 &	18.499880 &	0.0000000 \\
oera.L	& 0.1413504	& 0.0477646 &	2.959316 &	0.0033001 \\
oera.Q	& -0.2503659	& 0.0429312 &	-5.831790 &	0.0000000 \\
      \hline
    \end{tabularx}
\end{table}


\begin{table}[H]% Fix the table captions to sit directly on the table, but figures do NOT sit directly on the figure.
     \captionof{table}{Estimates from GLS for total energy ratio.}\label{first}
    \begin{tabularx}{\textwidth}{XXXXXX}
      \hline
        Timeperiod &	emmean &	SE &	df &	lower.CL &	upper.CL\\
      \hline
       1988-1997 &	0.2995118 &	0.0475806 &	36.19943 &	0.2030323 &	0.3959913 \\
1997-2010 &	0.7060960 &	0.0419773 &	38.51943 &	0.6211550 &	0.7910369 \\
2010-2020 &	0.4994115 &	0.0480066 &	37.62774 &	0.4021956 &	0.5966274 \\
      \hline
    \end{tabularx}
\end{table}



\begin{table}[H]% Fix the table captions to sit directly on the table, but figures do NOT sit directly on the figure.
     \captionof{table}{Contrasts from GLS for total energy ratio.}\label{first}
    \begin{tabularx}{\textwidth}{XXXXXX}
      \hline
        Comparison &	estimate &	SE &	df &	t.ratio &	p.value\\
      \hline 
1988-1997 - 1997-2010 &	-0.4065842 &	0.0623398 &	40.51631 &	-6.522060 &	0.0000 \\
1988-1997 - 2010-2020 &	-0.1998997 &	0.0675493 &	37.12310 &	-2.959316 &	0.0144 \\
1997-2010 - 2010-2020 &	0.2066845 &	0.0626456 &	41.44768 &	3.299267 &	0.0056 \\
      \hline
     \end{tabularx}
\end{table}

\begin{table}[H]
\captionof{table}{Model comparison for kangaroo rat proportional energy use.}\label{first}
\begin{tabularx}{\textwidth}{XX}
\hline 
Model specification    & AIC \\ 
\hline
intercept + timeperiod &	258.3581\\
intercept &	280.8497 \\
\hline
\end{tabularx}
\end{table}


\begin{table}[H]% Fix the table captions to sit directly on the table, but figures do NOT sit directly on the figure.
     \captionof{table}{Coefficients from GLM on kangaroo rat energy use.}\label{first}
    \begin{tabularx}{\textwidth}{XXXXX}
     \hline
       & Estimate & Std. Error & z.value & $Pr(>|z|)$ \\
     \hline
(Intercept) &	1.4032480 &	0.1503201	& 9.335068 &	0.0000000 \\
oera.L &	-1.1000833 &	0.2871738 &	-3.830723	& 0.0001278 \\
oera.Q &	0.5855493 &	0.2304516 &	2.540878	& 0.0110574 \\
      \hline
    \end{tabularx}
\end{table}


\begin{table}[H]% Fix the table captions to sit directly on the table, but figures do NOT sit directly on the figure.
     \captionof{table}{Estimates from GLM on kangaroo rat energy use.}\label{first}
    \begin{tabularx}{\textwidth}{XXXXXX}
      \hline
Timeperiod &	prob &	SE &	df &	asymp.LCL &	asymp.UCL  \\
\hline
1988-1997 &	0.9183528 &	0.0256462 &	Inf &	0.8680872 &	0.9686183 \\
1997-2010 &	0.7160901 &	0.0398537 &	Inf	& 0.6379782 &	0.7942020 \\
2010-2020 &	0.7035835 &	0.0456677 &	Inf	& 0.6140765 &	0.7930905 \\
\hline
    \end{tabularx}
\end{table}




\begin{table}[H]% Fix the table captions to sit directly on the table, but figures do NOT sit directly on the figure.
     \captionof{table}{Contrasts from GLM on kangaroo rat energy use.}\label{first}
    \begin{tabularx}{\textwidth}{XXXXXX}
      \hline
Comparison &	estimate &	SE &	df &	z.ratio &	p.value \\
\hline
1988-1997 - 1997-2010	& 1.4950249	& 0.3942281	& Inf	& 3.7922836 &	0.0004 \\
1988-1997 - 2010-2020	& 1.5557527	& 0.4061251	& Inf	& 3.8307227 &	0.0004 \\
1997-2010 - 2010-2020	& 0.0607279	& 0.2938992	& Inf	& 0.2066282 &	0.9767 \\
\hline
     \end{tabularx}
\end{table}



\begin{table}[H]% Fix the table captions to sit directly on the table, but figures do NOT sit directly on the figure.
     \captionof{table}{ Model comparison for \textit{C. baileyi} proportional energy use.}\label{first}
    \begin{tabularx}{\textwidth}{XX}
      \hline
      Model specification & AIC \\
      \hline
    intercept + timeperiod + treatment + timeperiod:treatment &	237.7643 \\
intercept + timeperiod + treatment	& 231.0963 \\
intercept + timeperiod &	460.8477 \\
intercept &	541.3799 \\
      \hline
    \end{tabularx}
\end{table}



\begin{table}[H]% Fix the table captions to sit directly on the table, but figures do NOT sit directly on the figure.
     \captionof{table}{Coefficients from GLM on \textit{C. baileyi} energy use.}\label{first}
    \begin{tabularx}{\textwidth}{XXXXX}
     \hline
       & Estimate & Std. Error & z.value & $Pr(>|z|)$ \\
     \hline
     (Intercept) &	-1.574028 &	0.1670168 &	-9.424368 &	0 \\
oera.L	& -1.409273 &	0.2010398 &	-7.009921 &	0 \\
oplottype.L	 & 2.184896 &	0.2267112 &	9.637355 &	0 \\
      \hline
    \end{tabularx}
\end{table}


\begin{table}[H]% Fix the table captions to sit directly on the table, but figures do NOT sit directly on the figure.
     \captionof{table}{Estimates from GLM on \textit{C. baileyi} energy use.}\label{first}
    \begin{tabularx}{\textwidth}{XXXXXXX}
      \hline
Timeperiod & Treatment &	prob &	SE &	df &	asymp.LCL &	asymp.UCL  \\
\hline
1997-2010 &	Control &	0.1069314 &	0.0258894 &	Inf &	0.0561890 &	0.1576737 \\
1997-2010 &	Exclosure &	0.7246076 &	0.0385129 &	Inf &	0.6491236 &	0.8000915 \\
2010-2020 &	Control &	0.0160560 &	0.0058224 &	Inf &	0.0046444 &	0.0274676 \\
2010-2020 &	Exclosure &	0.2639419 &	0.0428458 &	Inf &	0.1799657 &	0.3479181 \\
\hline
    \end{tabularx}
\end{table}




\begin{table}[H]% Fix the table captions to sit directly on the table, but figures do NOT sit directly on the figure.
     \captionof{table}{Contrasts from GLM on \textit{C. baileyi} energy use.}\label{first}
    \begin{tabularx}{\textwidth}{XXXXXXX}
      \hline
Comparison & Treatment &	estimate &	SE &	df &	z.ratio &	p.value \\
\hline
1997-2010 - 2010-2020 &	Control &	1.993013 &	0.2843132 &	Inf	& 7.009921 &	0 \\
1997-2010 - 2010-2020 &	Exclosure &	1.993013 &	0.2843132 &	Inf &	7.009921 &	0 \\
\hline
 \end{tabularx}
\end{table}


\section{Biomass analysis}

\subsection{Compensation}

We fit a generalized least squares (of the form $compensation = timeperiod$; note that “timeperiod” is coded as “oera” throughout) using the $gls$ function from the R package $nlme$ \cite{pinheiro2020}. Because values from monthly censuses within each time period are subject to temporal autocorrelation, we included a continuous autoregressive temporal autocorrelation structure of order 1 (using the $CORCAR1$ function). We compared this model to models fit without the autocorrelation structure and without the time period term using AIC. The model with both the time period term and the autocorrelation structure was the best-fitting model via AIC, and we used this model to calculate estimates and contrasts using the package \textit{emmeans} \cite{lenth2021}.

\subsection{Total biomass ratio}

As for compensation, we fit a generalized least squares of the form \textit{total energy ratio = timeperiod}, accounting for temporal autocorrelation between monthly censuses within each time period using a continuous autoregressive autocorrelation structure of order 1. We compared this model to models fit without the timeperiod term and/or autocorrelation structure, and found the full (timeperiod plus autocorrelation) model had the best performance via AIC. We used this model for estimates and contrasts.

\subsection{\textit{Dipodomys} proportional biomass}


Proportional biomass is bounded 0-1 and cannot be fit with generalized least squares. We therefore used a binomial generalized linear model of the form \textit{Dipodomys proportional biomass = timeperiod}. We compared a model fit with a timeperiod term to an intercept-only (null) model using AIC, and found the timeperiod term improved model fit. We used this model for estimates and contrasts.

Note that we were unable to incorporate temporal autocorrelation into generalized linear models, and we prioritized fitting models of the appropirate family over accounting for autocorrelation. Due to the pronounced differences between time periods for these variables, we were comfortable proceeding without explicitly accounting for autocorrelation.

\subsection{\textit{C. baileyi} proportional biomass}


As for kangaroo rat proportional biomass, we used a binomial generalized linear model to compare \textit{C. baileyi} proportional biomass across time periods. Because \textit{C. baileyi} occurs on both control and exclosure plots, we investigated whether the dynamics of \textit{C. baileyi}’s proportional biomass differed between treatment types. We compared models incorporating separate slopes, separate intercepts, or no terms for treatment modulating the change in \textit{C. baileyi} proportional biomass across time periods, i.e. comparing the full set of models:
  
  -	\textit{c baileyi proportional biomass = timeperiod + treatment + timeperiod:treatment}

-	\textit{c baileyi proportional biomass = timeperiod + treatment}

-	\textit{c baileyi proportional biomass = timeperiod}
We also tested a null (intercept-only) model of no change across time periods:
  
  - \textit{c baileyi proportional biomass = 1}

We found that the best-fitting model incorporated effects for time period and for treatment, but no interaction between them (\textit{c baileyi proportional biomass = timeperiod + treatment}). We therefore proceeded with this model.

\subsection{Tables}

\begin{table}[H]% Fix the table captions to sit directly on the table, but figures do NOT sit directly on the figure.
\captionof{table}{Model comparison for compensation.}\label{first}
\begin{tabularx}{\textwidth}{XX}
\hline
Model specification & AIC \\
\hline
intercept + timeperiod + autocorrelation	&-17.623354\\
intercept + autocorrelation	&-3.297103 \\
intercept + timeperiod &	92.184205 \\
intercept	& 207.804481\\
\hline
\end{tabularx}
\end{table}


\begin{table}[H]% Fix the table captions to sit directly on the table, but figures do NOT sit directly on the figure.
\captionof{table}{Coefficients from GLS model for compensation.}\label{first}
\begin{tabularx}{\textwidth}{XXXXX}
\hline
& Value & Std.Error &  t-value & p-value \\
\hline
(Intercept) &	0.3081443 &	0.0290539 &	10.605950 &	0.0000000 \\
oera.L &	0.0711412 &	0.0514131 &	1.383719 &	0.1673549 \\
oera.Q &	-0.2799121 &	0.0465252 &	-6.016352 &	0.000000 \\

\hline
\end{tabularx}
\end{table}

\begin{table}[H]% Fix the table captions to sit directly on the table, but figures do NOT sit directly on the figure.
\captionof{table}{Estimates from GLS for compensation.}\label{first}
\begin{tabularx}{\textwidth}{XXXXXX}
\hline
Timeperiod &	emmean &	SE &	df &	lower.CL &	upper.CL\\
\hline
1988-1997 & 0.1435663 & 0.0511419 & 39.28312 & 0.0401458 & 0.2469867 \\
1997-2010 & 0.5366915 & 0.0452745 & 41.91562 & 0.4453185 & 0.6280646 \\
2010-2020 & 0.2441751 & 0.0517205 & 41.17937 & 0.1397373 & 0.3486130 \\
\hline
\end{tabularx}
\end{table}


\begin{table}[H]% Fix the table captions to sit directly on the table, but figures do NOT sit directly on the figure.
\captionof{table}{Contrasts from GLS for compensation.}\label{first}
\begin{tabularx}{\textwidth}{XXXXXX}
\hline
Comparison &	estimate &	SE &	df &	t.ratio &	p.value \\
\hline
1988-1997 - 1997-2010 & -0.3931253 & 0.0673811 & 43.22895 & -5.834358 & 0.0000 \\
1988-1997 - 2010-2020 & -0.1006089 & 0.0727090 & 40.36882 & -1.383719 & 0.3588 \\
1997-2010 - 2010-2020 & 0.2925164 & 0.0678003 & 44.43055 & 4.314383 & 0.0003 \\
\hline
\end{tabularx}
\end{table}


\begin{table}[H]% Fix the table captions to sit directly on the table, but figures do NOT sit directly on the figure.
\captionof{table}{Model comparison for total biomass.}\label{first}
\begin{tabularx}{\textwidth}{XX}
\hline
Model specification & AIC \\
\hline
intercept + timeperiod + autocorrelation & -176.57761 \\
intercept + autocorrelation & -162.61339 \\
intercept + timeperiod & -15.98438 \\ 
intercept & 146.61442 \\
\hline
\end{tabularx}
\end{table}


\begin{table}[H]% Fix the table captions to sit directly on the table, but figures do NOT sit directly on the figure.
\captionof{table}{Coefficients from GLS for total biomass ratio.}\label{first}
\begin{tabularx}{\textwidth}{XXXXXX}
\hline
& Value & Std.Error  & t-value & p-value \\
\hline
(Intercept) & 0.4553971 & 0.0272418 & 16.716827 & 0.0000000 \\
oera.L & 0.1454493 & 0.0477989 & 3.042941 & 0.0025257 \\
oera.Q & -0.2531409 & 0.0427343 & -5.923594 & 0.0000000 \\
\hline
\end{tabularx}
\end{table}


\begin{table}[H]% Fix the table captions to sit directly on the table, but figures do NOT sit directly on the figure.
\captionof{table}{Estimates from GLS for total biomass ratio.}\label{first}
\begin{tabularx}{\textwidth}{XXXXXX}
\hline
Timeperiod &	emmean &	SE &	df &	lower.CL &	upper.CL\\
\hline
1988-1997 & 0.2492046 & 0.0476584 & 33.82432 & 0.1523326 & 0.3460765 \\
1997-2010 & 0.6620857 & 0.0419515 & 35.98516 & 0.5770030 & 0.7471684 \\
2010-2020 & 0.4549009 & 0.0480215 & 34.98703 & 0.3574107 & 0.5523911 \\

\hline
\end{tabularx}
\end{table}



\begin{table}[H]% Fix the table captions to sit directly on the table, but figures do NOT sit directly on the figure.
\captionof{table}{Contrasts from GLS for total biomass ratio.}\label{first}
\begin{tabularx}{\textwidth}{XXXXXX}
\hline
Comparison &	estimate &	SE &	df &	t.ratio &	p.value\\
\hline 
1988-1997 - 1997-2010 & -0.4128811 & 0.0621739 & 38.42746 & -6.640747 & 0.0000 \\
1988-1997 - 2010-2020 & -0.2056963 & 0.0675979 & 34.67694 & -3.042941 & 0.0121 \\
1997-2010 - 2010-2020 & 0.2071848 & 0.0624325 & 39.20390 & 3.318542 & 0.0054 \\
\hline
\end{tabularx}
\end{table}

\begin{table}[H]
\captionof{table}{Model comparison for kangaroo rat proportional biomass.}\label{first}
\begin{tabularx}{\textwidth}{XX}
\hline 
Model specification    & AIC \\ 
\hline
intercept + timeperiod & 215.2069 \\
intercept & 227.9608 \\
\hline
\end{tabularx}
\end{table}


\begin{table}[H]% Fix the table captions to sit directly on the table, but figures do NOT sit directly on the figure.
\captionof{table}{Coefficients from GLM on kangaroo rat biomass.}\label{first}
\begin{tabularx}{\textwidth}{XXXXX}
\hline
& Estimate & Std. Error & z.value & $Pr(>|z|)$ \\
\hline
(Intercept) & 1.6149566 & 0.1644937 & 9.817741 & 0.0000000 \\
oera.L & -1.1672395 & 0.3180813 & -3.669626 & 0.0002429 \\
oera.Q & 0.6619048 & 0.2473324 & 2.676175 & 0.0074468 \\
\hline
\end{tabularx}
\end{table}


\begin{table}[H]% Fix the table captions to sit directly on the table, but figures do NOT sit directly on the figure.
\captionof{table}{Estimates from GLM on kangaroo rat biomass.}\label{first}
\begin{tabularx}{\textwidth}{XXXXXX}
\hline
Timeperiod &	prob &	SE &	df &	asymp.LCL &	asymp.UCL  \\
\hline
1988-1997 & 0.9376458 & 0.0226460 & Inf & 0.8932605 & 0.9820310 \\
1997-2010 & 0.7454543 & 0.0385025 & Inf & 0.6699909 & 0.8209177 \\
2010-2020 & 0.7426552 & 0.0437171 & Inf & 0.6569713 & 0.8283392 \\
\hline
\end{tabularx}
\end{table}




\begin{table}[H]% Fix the table captions to sit directly on the table, but figures do NOT sit directly on the figure.
\captionof{table}{Contrasts from GLM on kangaroo rat biomass.}\label{first}
\begin{tabularx}{\textwidth}{XXXXXX}
\hline
Comparison &	estimate &	SE &	df &	z.ratio &	p.value \\
\hline
1988-1997 - 1997-2010 & 1.6360275 & 0.4372643 & Inf & 3.741508 & 0.0005 \\
1988-1997 - 2010-2020 & 1.6507259 & 0.4498349 & Inf & 3.669626 & 0.0007 \\
1997-2010 - 2010-2020 & 0.0146984 & 0.3057707 & Inf & 0.048070 & 0.9987 \\
\hline
\end{tabularx}
\end{table}



\begin{table}[H]% Fix the table captions to sit directly on the table, but figures do NOT sit directly on the figure.
\captionof{table}{ Model comparison for \textit{C. baileyi} proportional biomass.}\label{first}
\begin{tabularx}{\textwidth}{XX}
\hline
Model specification & AIC \\
\hline
intercept + timeperiod + treatment + timeperiod:treatment & 237.6847 \\
intercept + timeperiod + treatment & 231.2374 \\
intercept + timeperiod & 466.4937 \\
intercept + treatment & 346.2154 \\
intercept & 543.7811 \\
\hline
\end{tabularx}
\end{table}



\begin{table}[H]% Fix the table captions to sit directly on the table, but figures do NOT sit directly on the figure.
\captionof{table}{Coefficients from GLM on \textit{C. baileyi} biomass.}\label{first}
\begin{tabularx}{\textwidth}{XXXXX}
\hline
& Estimate & Std. Error & z.value & $Pr(>|z|)$ \\
\hline
(Intercept) & -1.538798 & 0.1671239 & -9.207525 & 0 \\
oera.L & -1.403286 & 0.2006948 & -6.992140 & 0 \\
oplottype.L & 2.270657 & 0.2298594 & 9.878462 & 0 \\
\hline
\end{tabularx}
\end{table}


\begin{table}[H]% Fix the table captions to sit directly on the table, but figures do NOT sit directly on the figure.
\captionof{table}{Estimates from GLM on \textit{C. baileyi} biomass.}\label{first}
\begin{tabularx}{\textwidth}{XXXXXXX}
\hline
Timeperiod & Treatment &	prob &	SE &	df &	asymp.LCL &	asymp.UCL  \\
\hline
1997-2010 & Control & 0.1041331 & 0.0255800 & Inf & 0.0539971 & 0.1542691 \\
1997-2010 & Exclosure & 0.7425132 & 0.0376727 & Inf & 0.6686761 & 0.8163504 \\
2010-2020 & Control & 0.0157248 & 0.0057341 & Inf & 0.0044861 & 0.0269634 \\
2010-2020 & Exclosure & 0.2838438 & 0.0439192 & Inf & 0.1977637 & 0.3699240 \\
\hline
\end{tabularx}
\end{table}




\begin{table}[H]% Fix the table captions to sit directly on the table, but figures do NOT sit directly on the figure.
\captionof{table}{Contrasts from GLM on \textit{C. baileyi} biomass.}\label{first}
\begin{tabularx}{\textwidth}{XXXXXXX}
\hline
Comparison & Treatment &	estimate &	SE &	df &	z.ratio &	p.value \\
\hline
1997-2010 - 2010-2020 & Control & 1.984546 & 0.2838253 & Inf & 6.99214 & 0 \\
1997-2010 - 2010-2020 & Exclosure & 1.984546 & 0.2838253 & Inf & 6.99214 & 0 \\
\hline
\end{tabularx}
\end{table}

\section{Covariates of rodent community change}



\begin{figure}
\begin{center}
\addFigure{.6}{./graphics/appendixAF1.png}
\captionof{figure}[Changes in overall community energy use (A), NDVI (B), and local climate (C) surrounding the 2010 shift in rodent community composition.]{Changes in overall community energy use (A), NDVI (B), and local climate (C) surrounding the 2010 shift in rodent community composition. As documented in \cite{christensen2018}, the 2010 transition followed a period of low abundance community-wide (A) and low plant productivity (B). Since 2010, the site has experienced two periods of drought (C) interspersed with an unusually wet period. Total rodent energy use (A) is calculated as the total energy use of all granviores on control plots ($Etot_C$) in each census period. The anomaly (shown) is calculated as the difference between the total energy use in each census period and the long-term mean of total energy use. Vertical dashed lines mark the dates of major transitions in the rodent community. NDVI anomaly (B) is calculated as the difference between monthly NDVI and the long-term mean for that month. NDVI data were obtained from Landsat 5, 7, and 8 using the \textit{ndvi} function in the R package \textit{portalr} \cite{maesk2006, vermote2016, christensen2019a}. Drought (C) was calculated using a 12-month Standardized Precipitation Evapotranspiraiton index (SPEI) for all months from 1989-2020, using the Thornthwaite method to estimate potential evapotranspiration (using the R package \textit{SPEI}; \cite{begueria2017, slette2019, cardenas2021}). Values greater than 0 (blue) indicate wetter than average conditions, and values less than 0 (red) indicate drier conditions. Values between -1 and 1 (horizontal lines) are considered within normal variability for a system, while values < -1 constitute drought \cite{slette2019}.}
\end{center}
\end{figure}


\chapter{Supplemental figures and tables for Chapter 3}

\section{Results for routes with complete temporal sampling}

Figures and tables from the main analysis, restricted to 199 routes with perfect temporal coverage (i.e. no missing time steps). All results are qualitatively the same as for the main analysis (739 routes, with a minimum of 27 of 30 time steps sampled for each route).

\subsection{Figures}

\begin{figure}[H]
\begin{center}
\addFigure{.6}{./graphics/appendixBF1.png} 
\captionof{figure}[Direction and magnitude of long-term trends]{Histograms showing the direction and magnitude of long-term trends for the individuals-driven (null-model; left) and observed (right) changes in biomass (A) and energy use (B), for communities with a significant slope and/or interaction term (for biomass, 141/199 routes; for energy use, 137/199 routes; Table B-1). Change is summarized as the ratio of the fitted value for the last year in the time series to the fitted value for the first year in the timeseries from the best-fitting model for that community. Values greater than 1 (vertical black line) indicate increases in total energy or biomass over time, and less than 1 indicate decreases. The individuals-driven dynamics (left) reflect the trends fit for the null model, while the observed dynamics (right) reflect trends incorporating both change in total abundance and change in the size structure over time. For communities best-described by syndromes of “coupled trends” or “no directional change”, the “individuals-driven” and “observed” ratios will be the same; for communities with “decoupled trends”, there will be different ratios for or “individuals-driven” and “observed” dynamics.

Among routes with temporal trends (“coupled trends” or “decoupled trends”), there are qualitatively different continental-wide patterns in individuals-driven and observed dynamics for total biomass and total energy use. 76\% of trends in individuals-driven (null model) dynamics for energy use are decreasing, and 72\% for biomass (Table B-2). For biomass, observed dynamics are balanced evenly between increases (50\% of routes) and decreases (50\%) - indicating that changes in the size structure produce qualitatively different long-term trends for biomass than would be expected given abundance changes alone. However, trends for energy use (which scales nonlinearly with biomass) are dominated by decreases (69\% of routes), more closely mirroring the trends expected given changes in individual abundance alone.
}
\end{center}
\end{figure}




\begin{figure}[H]
\begin{center}
\addFigure{.6}{./graphics/appendixBF2.png} 
\captionof{figure}[Compositional change]{Histograms of (A) change in mean body size from the first to the last five years of monitoring, (B) overall change in the size structure, and (C) change in species composition for routes whose dynamics for total biomass were best-described using no temporal trend (bottom row; intercept-only model), separate trends for observed and individuals-driven dynamics (middle row), or the same trend for observed and individuals-driven dynamics (top row). Change in mean body size (A) is calculated as the ratio of the mean body size of all individuals observed in the last 5 years of the timeseries relative to the mean body size of all individuals observed in the first 5 years. Overall change in the ISD (B) is calculated as the degree of turnover between the ISDs for the first and last five years of the timeseries (see text). Change in species composition (C) is Bray-Curtis dissimilarity comparing species composition in the first five years to the last five years.
}
\end{center}
\end{figure}

\subsection{Tables}


\begin{table}[H]
\caption[A proper table caption location]{The number and proportion of routes whose dynamics for total biomass and total energy use are best described by the following syndromes: no directional change (intercept-only model, \textit{biomass = 1} or \textit{energy use = 1}); a coupled trend (\textit{biomass = year} or \textit{energy use = year}); or a model with decoupled temporal trends for observed and individuals-driven dynamics (\textit{biomass = year * dynamics} or \textit{energy use = year * dynamics}, where \textit{dynamics} refers to observed or null model, individuals-driven dynamics).

}
%\begin{center}
\begin{tabularx}{\textwidth}{XXXX}
\hline Currency & Selected model & Number of routes & Proportion of routes \\
\hline
Total biomass & Intercept-only & 58 & 0.29 \\
Total biomass & Trend, not decoupled & 86 & 0.43 \\
Total biomass & Decoupled trend & 55 & 0.28 \\
Total energy use & Intercept-only & 62 & 0.31 \\
Total energy use & Trend, not decoupled & 115 & 0.58 \\
Total energy use & Decoupled trend & 22 & 0.11 \\
\hline
\end{tabularx}
%\end{center}

\end{table}



\begin{table}[H]
\caption[A proper table caption location]{The proportion of trends that are increasing (specifically, for which the ratio of the last fitted value to the first fitted value > 1) for individuals-driven and observed dynamics, for routes exhibiting temporal trends (“coupled trends” or “decoupled trends”) in total biomass and total energy use. Trends that are not increasing are decreasing.
}
%\begin{center}
\begin{tabularx}{\textwidth}{XXXX}
\hline Currency & Proportion of increasing individuals-driven trends & Proportion of increasing observed trends & Number of routes with temporal trends \\

\hline
Total biomass & 0.28 & 0.50 & 141 \\
Total energy use & 0.24 & 0.31 & 137 \\
\hline
\end{tabularx}
%\end{center}

\end{table}



\begin{table}[H]
\caption[A proper table caption location]{ANOVA table comparing ordinary linear models of the form \textit{absolute log ratio = syndrome} and \textit{absolute log ratio = 1}.
}
%\begin{center}
\begin{tabularx}{\textwidth}{XXXXXX}
\hline Res.Df & RSS & Df & Sum of Sq & F & $Pr(>F)$ \\

\hline
196 & 5.993024 & NA & NA & NA & NA \\
198 & 11.105203 & -2 & -5.11218 & 83.59613 & 0 \\

\hline
\end{tabularx}
%\end{center}

\end{table}



\begin{table}[H]
\caption[A proper table caption location]{Estimates (calculated using \textit{emmeans} \cite{lenth2021}) for the mean absolute log ratio of mean mass for routes whose dynamics for biomass were best-described by different syndromes of change.
}
%\begin{center}
\begin{tabularx}{\textwidth}{XXXXXX}
\hline categorical fit & emmean & SE & df & lower.CL & upper.CL \\
\hline
Coupled trend & 0.2084997 & 0.0188558 & 196 & 0.1713133 & 0.2456861 \\
Decoupled trends & 0.5779023 & 0.0235784 & 196 & 0.5314024 & 0.6244021 \\
No trend & 0.2385438 & 0.0229605 & 196 & 0.1932625 & 0.2838251 \\

\hline
\end{tabularx}
%\end{center}

\end{table}



\begin{table}[H]
\caption[A proper table caption location]{Contrasts for absolute log ratio of mean mass, calculated using \textit{emmeans} \cite{lenth2021}.
}
%\begin{center}
\begin{tabularx}{\textwidth}{XXXXXX}
\hline contrast & estimate & SE & df & t.ratio & p.value \\
\hline
Coupled trend - Decoupled trends & -0.3694026 & 0.0301908 & 196 & -12.235620 & 0.0000000 \\
Coupled trend - No trend & -0.0300441 & 0.0297107 & 196 & -1.011221 & 0.5706639 \\
Decoupled trends - No trend & 0.3393585 & 0.0329108 & 196 & 10.311453 & 0.0000000 \\

\hline
\end{tabularx}
%\end{center}

\end{table}



\begin{table}[H]
\caption[A proper table caption location]{ANOVA table comparing binomial generalized linear models of the form \textit{ISD turnover = syndrome} and \textit{ISD turnover = 1}.
}
%\begin{center}
\begin{tabularx}{\textwidth}{XXXXXX}
\hline Resid. Df & Resid. Dev & Df & Deviance & $Pr(>Chi)$ \\

\hline
196 & 4.053082 & NA & NA & NA \\
198 & 4.253015 & -2 & -0.1999328 & 0.9048678 \\


\hline
\end{tabularx}
%\end{center}

\end{table}


\begin{table}[H]
\caption[A proper table caption location]{ANOVA table comparing binomial generalized linear models of the form \textit{Bray-Curtis dissimilarity = syndrome} and \textit{Bray-Curtis dissimilarity = 1}.
}
%\begin{center}
\begin{tabularx}{\textwidth}{XXXXXX}
\hline Resid. Df & Resid. Dev & Df & Deviance & $Pr(>Chi)$ \\

\hline
196 & 4.455349 & NA & NA & NA \\
198 & 5.044039 & -2 & -0.5886892 & 0.7450197 \\

\hline
\end{tabularx}
%\end{center}

\end{table}


\section{Statistical comparisons of distributions in Figure 3-4}


\begin{table}[H]
\caption[A proper table caption location]{ANOVA table comparing ordinary linear models of the form \textit{absolute log ratio = syndrome} and \textit{absolute log ratio = 1}. The fit incorporating syndrome is superior to the intercept-only model ($p < 0.0001$).
}
%\begin{center}
\begin{tabularx}{\textwidth}{XXXXXX}
\hline Res.Df & RSS & Df & Sum of Sq & F & $Pr(>F)$ \\

\hline
736 & 20.81159 & NA & NA & NA & NA \\
738 & 35.42466 & -2 & -14.61307 & 258.395 & 0 \\

\hline
\end{tabularx}
%\end{center}

\end{table}



\begin{table}[H]
\caption[A proper table caption location]{Estimates (calculated using \textit{emmeans} \cite{lenth2021}) for the mean absolute log ratio of mean mass for routes whose dynamics for biomass were best-described by different syndromes of change. Routes with decoupled long-term trends between biomass and individuals-driven dynamics have higher absolute log ratios (mean .56, 95\% credible interval .53-.58) than routes with covarying trends in biomass and individual abundance (mean of .2; 95\% interval .18-.22) or no detectable temporal trend (mean of .22; .2-.24).
}
%\begin{center}
\begin{tabularx}{\textwidth}{XXXXXX}
\hline categorical fit & emmean & SE & df & lower.CL & upper.CL \\
\hline
Coupled trend & 0.2007265 & 0.0089755 & 736 & 0.1831058 & 0.2183472 \\
Decoupled trends & 0.5587675 & 0.0137759 & 736 & 0.5317228 & 0.5858123 \\
No trend & 0.2211238 & 0.0108771 & 736 & 0.1997699 & 0.2424777 \\
\hline
\end{tabularx}
%\end{center}

\end{table}



\begin{table}[H]
\caption[A proper table caption location]{Contrasts for absolute log ratio of mean mass, calculated using \textit{emmeans} \cite{lenth2021}.There is a significant contrast between routes with decoupled trends and the other two syndromes of dynamics (both contrasts, $p < 0.001$), but not between routes showing the “no trend” and “coupled trend” syndromes (contrast $p = .31$).
}
%\begin{center}
\begin{tabularx}{\textwidth}{XXXXXX}
\hline contrast & estimate & SE & df & t.ratio & p.value \\
\hline
Coupled trend - Decoupled trends & -0.3580410 & 0.0164419 & 736 & -21.776134 & 0.0000000 \\
Coupled trend - No trend & -0.0203973 & 0.0141022 & 736 & -1.446391 & 0.3176979 \\
Decoupled trends - No trend & 0.3376437 & 0.0175524 & 736 & 19.236285 & 0.0000000 \\
\hline
\end{tabularx}
%\end{center}

\end{table}



\begin{table}[H]
\caption[A proper table caption location]{ANOVA table comparing binomial generalized linear models of the form \textit{ISD turnover = syndrome} and \textit{ISD turnover = 1}.  The model incorporating syndrome is not superior to the intercept-only model $(p = .9)$.
}
%\begin{center}
\begin{tabularx}{\textwidth}{XXXXXX}
\hline Resid. Df & Resid. Dev & Df & Deviance & $Pr(>Chi)$ \\

\hline
736 & 14.09312 & NA & NA & NA \\
738 & 14.28236 & -2 & -0.1892428 & 0.9097173 \\


\hline
\end{tabularx}
%\end{center}

\end{table}


\begin{table}[H]
\caption[A proper table caption location]{ANOVA table comparing binomial generalized linear models of the form \textit{Bray-Curtis dissimilarity = syndrome} and \textit{Bray-Curtis dissimilarity = 1}. The model incorporating syndrome is not superior to the intercept-only model ($p = .37$).
}
%\begin{center}
\begin{tabularx}{\textwidth}{XXXXXX}
\hline Resid. Df & Resid. Dev & Df & Deviance & $Pr(>Chi)$ \\

\hline
736 & 20.10447 & NA & NA & NA \\
738 & 22.11983 & -2 & -2.015363 & 0.3650643 \\
\hline
\end{tabularx}
%\end{center}

\end{table}

\chapter{Supplemental analyses for Chapter 4}

Test for third appendix file.


