
\chapter{Temporal changes in the individual size distribution decouple long-term trends in abundance, biomass, and energy use of North American breeding bird communities}

\section{Background}

Understanding the interrelated dynamics of size- and individuals-based dimensions of biological abundance is key to understanding trends in biodiversity in the Anthropocene. Total individual abundance - i.e. the total number of individual organisms present in a system - and size-based currencies, such as the total biomass or total metabolic flux (“energy use”) of a system, are intertwined, but nonequivalent, measures of biological function. Abundance is most closely tied to species-level population dynamics, while size-based metrics more directly reflect assemblage-level resource use and contributions to materials fluxes at the broader ecosystem scale \cite{morlon2009, dornelas2011, connolly2005, white2007}. While these currencies are naturally linked \cite{morlon2009, henderson2010}, changes in size composition can decouple the dynamics of one currency from another \cite{ernest2009, dornelas2011, white2004, white2007, yen2017}. This can mean that intuition from one currency may be misleading about others; a trend in numerical abundance might mask contrasting dynamics occurring with respect to biomass or total energy use \cite{white2004}. Changes in size composition strong enough to decouple currencies may be symptomatic of important changes in ecosystem status - e.g. abundance-biomass comparison curves in the aquatic realm \cite{petchey2010} or size-biased extinctions \cite{young2016, smith2018}. In the Anthropocene, as ecosystems globally undergo unprecedented levels of ecological and functional change \cite{fisher2010}, it is especially important to understand long-term trends in individual abundance, total biomass, total energy use, and the relationship between these currencies.

At the community scale, changes in the relationship between size and abundance can signal important shifts in community structure and functional composition. To the extent that size is a proxy for other functional traits, changes or consistency in the community-level size structure (individual size distribution, ISD) over time may reflect processes related to niche structure \cite{white2007, petchey2010}. Strong size shifts can decouple the relationship between abundance and biomass. This is especially well-established in aquatic systems, where such changes in the scaling between abundance and biomass often signal ecosystem degradation \cite{warwick1994, kerr2001, petchey2010}. Compensatory shifts in the size structure can buffer community function (in terms of biomass or energy use) against changes in abundance \cite{ernest2009, white2004, terry2015}. Or, consistency in the size structure may maintain the relationship between size- and -abundance based currencies, even as species composition, total abundance, and total biomass and total energy use fluctuate over time, which can reflect consistency in the niche structure over time \cite{holling1992}. In contrast to terrestrial trees and aquatic systems, which have received considerable attention in this regard \cite{kerr2001, white2007}, the relationships between community structure, total biomass, and individual abundance remain relatively unknown for terrestrial animals (but see \cite{white2004}). Terrestrial animal communities exhibit size structure \cite{thibault2011, ernest2005}, and case studies have demonstrated that size shifts can decouple the dynamics of individual abundance, biomass, and energy use for terrestrial animals \cite{white2004, yen2017}, but do not always do so \cite{hernandez2011}. Establishing generalities in these dynamics is especially pertinent in the Anthropocene, as these communities are experiencing extensive and potentially size-structured change, with implications at community, ecosystem, and global scales \cite{young2016,schmitz2018}.

Macroecological-scale synthesis on the interrelated dynamics of the ISD, total abundance, and community function for terrestrial animals has been constrained by 1) a lack of community-level size and abundance timeseries data for these systems \cite{thibault2011, white2007}, and 2) appropriate statistical methods for relating change in the size structure to changes in abundance and function \cite{thibault2011, yen2017}. In contrast to aquatic and forest systems, most long-term surveys of animal communities do not collect data on individuals’ sizes across a full community (with the exception of small mammal studies, which have made major contributions to our understanding of the dynamics of size, abundance, and function for these systems; \cite{white2004, ernest2005, hernandez2011, kelt2015}. Global, continental, or population-wide studies capture different phenomena \cite{white2007, mcgill2015}. The ISDs for terrestrial animals, and specifically for determinate growing taxa (e.g. mammals, birds), are often complex, multimodal distributions strongly determined by community species composition \cite{holling1992, thibault2011, ernest2005, yen2017}  and less statistically tractable than the power-law ISDs found in aquatic and tree systems \cite{kerr2001, white2007}. Quantifying change in the size structure, and relating this to change in community-wide abundance and function, is not as straightforward as computing and comparing slopes. As a result, we do not have a general understanding of either 1) how the size structures for these systems behave over time or 2) the extent to which changes in community size structure decouple the community-level dynamics of abundance, biomass, and energy use in these systems.

Here, we begin to address this gap by exploring how temporal changes in the size structure modulate the relationship between total abundance, energy, and biomass for communities of North American breeding birds. We used allometric scaling to estimate community size and abundance data for the North American Breeding Bird Survey, and evaluated how changes in total abundance, biomass, and energy use have co-varied from 1988-2018. Specifically, we examined: 1) How often do these currencies change together vs. have decoupled dynamics?; 2) What are the dominant directions and magnitudes of the overall change over time and degree of decoupling between the currencies?; 3) To what extent do changes in species composition and community size structure translate into decoupling in the temporal trends of different currencies at the community scale?

\section{Methods}

Code to replicate these analyses is available online at \hyperlink{https://github.com/diazrenata/diss-BBSsize}{https://github.com/diazrenata/diss-BBSsize} and \hyperlink{https://github.com/diazrenata/invisible-string}{https://github.com/diazrenata/invisible-string}. 

\subsection{Bird abundance data}

We used data from the Breeding Bird Survey \cite{pardieck2019a} to evaluate trends in abundance, biomass, and energy use. The Breeding Bird Survey consists of roughly 40 kilometer-long survey routes distributed throughout the United States and Canada. Routes are surveyed annually during the breeding season (predominately May-June), via 50 3-minute point counts during which all birds seen or heard are identified to species \cite{pardieck2019a}. Sampling began in 1966, and routes have been added over time to a current total of roughly 3000 routes \cite{pardieck2019a}). We take the route to be the “community” scale \cite{thibault2011}. We filtered the Breeding Bird Survey data to remove taxa that are poorly sampled through the point-count methods used in the Breeding Bird Survey, following \cite{harris2018}. We accessed the data, and performed this preliminary cleaning and filtering, using the R package \textit{MATSS} \cite{ye2020}.

We explored trends in abundance, biomass, and energy use over the 30-year time period from 1988-2018. We selected these years to provide a temporal window sufficient to detect long-term trends \cite{cusser2020}), while allowing for a substantial number of routes. To avoid irregularities caused by missing time steps, we restricted the main analysis to routes that had been sampled in at least 27 of 31 years in this window ($n = 739$), and compared these results to a more strict selection of routes that were sampled in every year ($n = 199$). Results for this more conservative subset of routes were qualitatively the same as for the more inclusive selection of routes (Appendix B). 

\subsection{Estimated size data}

The Breeding Bird Survey contains abundances for all species along a route in each year, but does not include measurements of individual body size. We generated body size estimates for individual birds assuming that intraspecific size distributions are normally distributed around a species’ mean body size (following \cite{thibault2011}). Using records of species’ mean and standard deviation body sizes from \cite{dunning2008}, we drew individuals’ body sizes from the appropriate normal distributions. For species for which there was not a standard deviation recorded in \cite{dunning2008} (185 species affected, of 421 total), we estimated the standard deviation using an allometric scaling relationship between mean and standard deviation in body mass based on the records for which the standard deviation was provided. Using these records, we fit a linear model of the form $\text{log}(\text{var}(m)) = \text{log}(\bar{m})$), where $m$ is body mass (model $R^2 = .89$), which yielded the scaling relationship $\text{var}(m)= 0.0047\bar{m}^{2.01}$ (see also \cite{thibault2011} for this scaling relationship calculated on a slightly different subset of the Breeding Bird Survey). For species with multiple records in \cite{dunning2008}, we used the mean $\bar{m}$ and standard deviation of $m$ across all records (averaging across sexes, subspecies, and records from different locations). We performed this averaging after estimating any missing standard deviation measurements. 

We simulated individual-level body mass measurements for each individual bird observed by drawing from the normal distribution with the mean and standard deviation of body mass estimated for each bird’s species \cite{thibault2011}. For each individual bird, we estimated metabolic rate as $10.5(m^{0.713})$ \cite{fristoe2015, nagy2005, mcnab2009}. For each route in each year, we computed total energy use, total biomass, and total abundance by summing over all individuals observed on that route in that year. This method does not incorporate intraspecific variation in body size across geographies or changes in species’ mean body size over time \cite{dunning2008, gardner2011}. However, it makes it possible to conduct macroecological studies of avian size distributions at a spatial and temporal scale that would otherwise be impossible \cite{thibault2011}. An R package containing code to generate these size estimates is available at \hyperlink{https://github.com/diazrenata/diss-BBSsize}{https://github.com/diazrenata/diss-BBSsize}.  

\subsection{Comparing abundance- and size- based currencies}

Comparing trends across different currencies is a nontrivial statistical problem. Because different currencies vary widely in their units of measure (e.g. abundance in the hundreds of individuals; total biomass in the thousands of grams), it is challenging to interpret differences in magnitude of slope across different currencies. Transformation and scaling using common approaches (such as a square-root transformation, or rescaling each currency to a mean of 0 and a standard deviation of 1) destroys information about the degree of variability within each currency that is necessary in order to make comparisons between currencies for the same timeseries.

Rather than attempting to compare slopes across currencies or to transform different currencies to a common scale, we used a simple null model to compare the observed dynamics for biomass and energy use to the dynamics that would occur in a scenario in which the species composition (and therefore, in this context, size structure) of the community was consistent throughout the timeseries, but in which total individual abundance varied over time consistent with the observed dynamics. For each route, we characterized the “observed” timeseries of total biomass and total energy use by simulating size measurements for all individuals observed in each time step and summing across individuals, using the method described above. We then simulated timeseries for “individuals-driven” dynamics of biomass and energy use incorporating observed changes in community-wide individual abundance over time, but under a scenario of consistent species (and therefore approximate size) composition over time. For each community, we characterized the timeseries-wide probability of an individual drawn at random from the community belonging to a particular species ($P(s_i)$) as each species’ mean relative abundance taken across all timesteps:

$P(s_i) = \frac{\sum_t^T{\frac{n_{i, t}}{N_T}}}{T}$

where $n_{i,t}$ is the abundance of species $i$ in timestep $t$, $N_t$ is the total abundance of all species in timestep $t$, and $T$ is the total number of timesteps. For each timestep t, we randomly assigned species’ identities to the total number of individuals (of all species) observed in that time step ($N_t$) by drawing with replacement from a multinomial distribution with probabilities weighted according to $P(s)$ for all species. We then simulated body size measurements for individuals, and calculated total energy use and total biomass, following the same procedure as for the observed community. This characterizes the dynamics for size-based currencies expected if the species (and size) composition of the community did not change over time, but incorporating observed fluctuations in total individual abundance. We refer to these dynamics as “individuals-driven” dynamics.

\subsection{Long-term trends}
For each route, we evaluated the 30-year trend in biomass (or energy use) and compared this to the trend derived from the “individuals-driven” null model using generalized linear models with a Gamma family and log link (appropriate for strictly-positive response variables such as biomass or total energy use). We selected between four model formulas, corresponding to qualitatively different “syndromes” of change, to characterize 1) the trend in biomass (or energy use) over time and 2) whether this trend deviates from the trend expected given only changes in individual abundance:

 - \textit{biomass = year * dynamics} or \textit{energy use = year * dynamics} in which “dynamics” refers to being either the “observed” or “individuals-driven” (null model) dynamics. This model fits a slope and intercept for the observed trend in biomass or energy use over time, and a separate slope and intercept for the trend drawn from the individuals-driven dynamics. We refer to this model as describing a syndrome of “Decoupled trends” between individuals-driven and observed dynamics.

- \textit{biomass = year + dynamics} or \textit{energy use = year + dynamics}. This model fits a separate intercept, but not slope, for the individuals-driven and observed dynamics. This model was never selected as the best-performing description of community dynamics. 

- \textit{biomass = year} or \textit{energy use = year}. This model fits a temporal trend, but does not fit separate trends for the observed and individuals-driven dynamics. We refer to this syndrome as “Coupled trends” between individuals-driven and observed dynamics.

- \textit{biomass = 1} or \textit{energy use = 1}. The intercept-only model describes no directional change over time for either the observed or individuals-driven dynamics, and we refer to this syndrome as describing “No directional change” for either type of dynamics. 


We selected the best-fitting model for each route using AICc. In instances where multiple models had AICc scores within two AICc units of the best-fitting model, we selected the simplest model within two AICc units of the best score.
We used the model predictions from each route’s best-fitting model to characterize the direction and slope of each route’s long term trends for individuals-driven (null) and biomass- or energy use-driven (observed) dynamics. For each route’s best-fitting model, we extracted the predicted values for the first (usually 1988) and last (usually 2018) year sampled, for both the observed and null trajectories. We calculated the magnitude of change over time as the ratio of the last (2018) to the first (1988) value, and characterized the direction of the long-term trend as increasing if this ratio was greater than one, and decreasing if it was less than one.

\subsection{Relating change in community structure to decoupling between individual- and size-based dynamics}

We used dissimilarity metrics to explore the extent to which change in community structure propagates to decoupling between long-term trends in individual abundance and total biomass and energy use. These dissimilarity metrics are most readily interpretable when making pairwise comparisons (as opposed to repeated comparisons over a timeseries). We therefore made comparisons between the first and last five-year intervals in each timeseries, resulting in a “begin” and “end” comparison separated by a relatively consistent window of time across routes (usually 19-20 years). The use of five-year periods corrects for sampling effects \cite{white2004a}), smooths out interannual variability, and, by including a relatively large proportion $\frac{1}{3}$ of the total timeseries, partially mitigates the impact of scenarios where the start and end values do not align with the long-term trend.

We calculated three metrics to explore how changes in community composition and size structure translate into decoupling between individuals-driven and observed dynamics for biomass and energy use. First, we evaluated the change in average community-wide body size, calculated as the absolute log ratio of mean body size in the last five years relative to the mean body size in the first five years:
$\text{absolute log ratio} = \abs{\ln{\frac{\bar{m}_{last5}}{\bar{m}_{first5}}}}$ where $\bar{m}_{first5}$ and $\bar{m}_{last5}$ is the mean body size of all individuals observed in the first and last 5 years, respectively. We used the absolute log ratio in order to focus on the magnitude, rather than the direction, of change in body size (see also \cite{supp2014} for the use of the absolute log ratio to examine the magnitudes of differences between values). Large changes in community-wide average body size are, by mathematical necessity, expected to translate into decoupling between observed and abundance-driven dynamics.

Second, we calculated measures of turnover in the size structure and in species composition. We calculated turnover in the ISD using a measure inspired by an overlap measure that has previously been applied to species body size distributions in mammalian communities \cite{read2018}. We characterized each “begin” or “end” ISD as a smooth probability density function by fitting a Gaussian mixture model (with up to 15 Gaussians, fit following \cite{thibault2011}) to the raw distribution of body masses, and extracting the fitted probability density at 1000 evaluation points corresponding to body masses encompassing and extending beyond the range of body masses present in this dataset (specifically, from 0 to 15 kilograms; mean body masses in this dataset range from 2.65 grams, for the Calliope hummingbird \textit{Selasphorus calliope}, to 8.45 kg, for the California condor \textit{Gymnogyps californianus}). We rescaled each density function such that the total probability density summed to 1. To calculate the degree of turnover between two ISDs, we calculated the area of overlap between the two density smooths as $\text{ISD turnover} = \sum{\text{min}(density1_i, density2_i)}$ where $density1_i$ is the probability density from the density smooth for the first ISD at evaluation point $i$, and $density2_i$ is the probability density from the density smooth for the second ISD at that evaluation point. We subtracted this quantity from 1 to obtain a measure of turnover between two ISDs. 

To evaluate turnover in species composition between the five-year time periods, we calculated Bray-Curtis dissimilarity between the two communities using the R package \textit{vegan} \cite{pinheiro2020}.

We tested whether routes whose dynamics were best-described by each “syndrome” of change – i.e. “Decoupled trends”, “Coupled trends”, or “No directional change” – differed in 1) the magnitude of change in mean body size; 2) turnover in the ISD over time; or 3) species compositional turnover (Bray-Curtis dissimilarity) over time. For change in mean body size, we fit an ordinary linear model of the form \textit{absolute log ratio = syndrome}. We compared this model to an intercept-only null model of the \textit{absolute log ratio = 1}. Because our metrics for turnover in the ISD and species composition are bounded from 0-1, we analyzed these metrics using binomial generalized linear models of the form \textit{ISD turnover = syndrome} and \textit{Bray Curtis dissimilarity = syndrome}, and again compared these models to intercept-only null models. In instances where the model fit with a term for syndrome outperformed the intercept-only model, we calculated model estimates and contrasts using the R package \textit{emmeans} \cite{lenth2021}.

\section{Results}
Of the 739 routes in this analysis, approximately 70\% (500/739 for biomass, and 509/739 for energy use) exhibited syndromes of “Decoupled trends” or “Coupled trends” (that is, were best-described using a model incorporating a temporal trend in individuals-driven and/or biomass or energy use-driven dynamics (Table 3-1)). All results were qualitatively the same using a subset of 199 routes with complete temporal sampling over time (Appendix B). Trends driven by individual abundance, as reflected by the dynamics of a simple null model, were strongly dominated by declines (335 decreases and 165 increases for individuals-driven dynamics in biomass, and 355 decreases and 154 increases for individuals-driven dynamics in energy use; Figure 3-2; Table 3-2). However, for biomass, the long-term temporal trends were evenly balanced between increases and decreases (256 decreasing trends, and 244 increasing trends; Figure 3-2; Table 3-2). For energy use, there was a greater representation of decreasing trends than for biomass, but still less so than for strictly abundance-driven dynamics (329 decreasing trends and 180 increasing trends; Figure 3-2; Table 3-2).

These divergent aggregate outcomes in individual abundance, energy use, and especially biomass occurred due to decoupling in the long-term trends for these different currencies. For a substantial minority of routes (20\% of all routes for biomass, and 7\% of all routes for energy use), long-term dynamics were best-described as a syndrome of “Decoupled trends” (that is, with a different slope for biomass or energy use-driven dynamics than for the “null”, individuals-driven, trend) (Table 3-1). When this decoupling occurred, it was dominated by instances in which the slope for individuals-driven dynamics was more negative than that for biomass or energy use (Figure 3-3).

Decoupling between the long-term trajectories of individual abundance and energy use or biomass is, by definition, indicative of some degree of change in the community size structure over time. Routes whose dynamics for biomass were best-described as syndromes of decoupled trends over time had a higher absolute log ratio of mean mass (i.e. greater magnitude of change, either increasing or decreasing, in mean mass over time) than routes with coupled or no directional trends (Figure 3-4; Appendix B). However, there was not a detectable difference in the degree of temporal turnover in the ISD overall (Figure 3-4; Appendix B), or in species composition (Figure 3-4; Appendix B), compared between routes that exhibited different syndromes of change.

\section{Discussion}

\subsection{Abundance, biomass, and energy use are nonequivalent currencies}

Simultaneously examining multiple currencies of community-level abundance revealed qualitatively different continent-wide patterns in the long-term trends for the number of individuals, total biomass, and energy use. While long-term trends in individual abundance were dominated by decreases, long-term trends in biomass were evenly split between increases and decreases, and trends in energy use were again dominated by declines (Figure 3-2). These different currencies, though intrinsically linked, describe nonequivalent dimensions of community function and reflect different classes of structuring processes \cite{morlon2009}. Individual abundance is most directly linked to species-level population dynamics of the type often considered in classic, particularly theoretical, approaches to studying competition, compensation, and coexistence (e.g. \cite{hubbell2001, chesson2000}). Biomass most directly reflects the productivity of a community and its potential contributions to materials fluxes in the broader ecosystem context, whereas energy use - by taking into account the metabolic inefficiencies of organisms of different body size - characterizes the total resource use of a community and may come the closest to capturing signals of bottom-up constraints, “Red Queen” effects, or zero-sum competitive dynamics \cite{vanvalen1973, ernest2008, ernest2009, morlon2009, white2004}. Our results underscore that, while trends in individual abundance, biomass, and energy use naturally co-vary to some extent, shifts in the community size structure can and do produce qualitatively different trends for these different currencies. These may reflect contrasting long-term changes in different types of community processes - for example, shifts in habitat structure that affect the optimal body sizes for organisms in a system, but do not result in overall changes in resource availability (e.g. \cite{white2004}). Moreover, extrapolating the long-term trend from one currency to another may mask underlying changes in the community that complicate these dynamics. To appropriately monitor different dimensions of biodiversity change, it is therefore important to focus on the specific currency most closely aligned with the types of processes and dynamics - e.g. population fluctuations, resource limitation, or materials fluxes - of interest in a particular context.

\subsection{For North American breeding birds, biomass has declined less than abundance or energy use}

For communities with a decoupling in the long-term trends of biomass, energy use, and abundance, this decoupling is indicative of a directional shift in the size structure of the community. For the communities of breeding birds across North America considered here, the long-term trends in total biomass are often less negative than trends in total individual abundance or total energy use (Figure 3). This consistent (but not ubiquitous) signal corresponds to community-level increases in average body size that partially or completely buffer changes in total biomass against declines in the overall number of individuals. This contrasts with taxonomically general, global concerns that larger-bodied species are more vulnerable to extinction and population declines than smaller ones \cite{young2016, dirzo2014, smith2018}. However, it is consistent with previous findings from the Breeding Bird Survey \cite{schipper2016}. The long-term trends for communities of different taxonomic groups, geographies, or temporal spans may show different effects related to different facets of global change and biodiversity responses. We also note that these increases in body size do not generally appear great enough to decouple the long-term trends in energy use from total individual abundance (Figure 3-3). Energy use scales nonlinearly with body size with an exponent less than 1, which means that community-wide increases in mean body size result in smaller increases in total energy use than in total biomass.

\subsection{Complex relationships between compositional change and community-level properties}

The decoupling between the long-term trends for biomass, individual abundance, and energy use demonstrated in many of the communities studied here is symptomatic of a directional shift in the size structure - in these instances, generally favoring larger bodied species. However, examining the community-wide dynamics of turnover in species composition and the overall size structure reveals that the relationship between changes in community structure and changes in the scaling between different currencies of community-wide abundance is considerably more nuanced than simple directional shifts in mean size. Routes that exhibit a statistically detectable decoupling between total biomass and total abundance show large changes in average body size compared to routes for which biomass and abundance either change more nearly in concert with each other or do not show temporal trends (Figure 3-4; Appendix B). This aligns naturally with mathematical intuition given the intrinsic relationship between average body size, total abundance, and total biomass. However, these routes are not extraordinary in terms of their overall degree of temporal turnover in either the size structure or in species composition. Rather, the levels of turnover in overall community size structure and species composition are comparable between routes that show decoupling between abundance and biomass, statistically indistinguishable trends, or no temporal trends in either currency (Figure 3-4; Appendix B).

For many communities, therefore, there has been appreciable change in species and size composition that does not manifest in a shift in the overall community-wide mean body size or mean metabolic rate sufficient to decouple the dynamics of biomass, abundance, and energy use. These changes may signal changes in functional composition equally important as the ones that manifest in directional shifts in community-wide average body size. For the complex, multimodal size distributions that are the norm for avian communities \cite{thibault2011}, changes in the number and position of modes may be as important as changes in higher-level statistical moments such as the overall mean. At present, the field lacks the statistical tools and conceptual frameworks to quantify and interpret these nuanced changes, especially at the macroecological scale of the current study \cite{thibault2011, yen2017}. However, this is an excellent opportunity for more system-specific work, informed by natural history knowledge and process-driven expectation, to characterize more nuanced changes in the size structure of specific communities and identify the underlying drivers of these changes. To facilitate these efforts in the context of the Breeding Bird Survey, the R package we have developed to characterize the individual size distributions for avian communities based on species’ identities and/or mean body sizes is freely available for re-use and wider applications (\hyperlink{https://github.com/diazrenata/diss-BBSsize}{https://github.com/diazrenata/diss-BBSsize}; the general-use version of this package is currently in development at \hyperlink{https://github.com/diazrenata/birdsize}{https://github.com/diazrenata/birdsize}).

\subsection{Conclusion}

This analysis demonstrates the current power, and limitations, of a data-driven macroecological perspective on the interrelated dynamics of community size structure and different dimensions of community-wide abundance for terrestrial animal communities. For breeding bird communities across North America, we find that changes in species and size composition produce qualitatively different aggregate patterns in the long-term trends of abundance, biomass, and energy use, highlighting the nuanced relationship between these related, but decidedly nonequivalent, currencies and reflecting widespread changes in community size structure that may signal substantive changes in functional composition. Simultaneously, the complex relationship between turnover in community species and size composition, and the scaling between different currencies of community-level abundance, highlights opportunities for synergies between recent computational and statistical advances, case studies grounded in empiricism and natural history, and future macroecological-scale synthesis to realize the full potential of this conceptual space.
